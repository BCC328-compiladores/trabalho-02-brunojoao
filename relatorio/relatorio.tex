\documentclass[12pt,a4paper]{article}
\usepackage[utf8]{inputenc}
\usepackage[T1]{fontenc}
\usepackage[brazil]{babel}
\usepackage{lmodern}
\usepackage{geometry}
\usepackage{hyperref}
\usepackage{enumitem}
\usepackage{parskip}

\geometry{left=3cm,right=2cm,top=3cm,bottom=2cm}
\setlist[itemize]{leftmargin=1.8em, itemsep=0.2em, topsep=0.3em}

\hypersetup{colorlinks=true, linkcolor=blue, urlcolor=blue, citecolor=blue}

\title{Relatórios do Projeto — Compilador SL (Entregas 1, 2 e 3)}
\author{Nome do Aluno \\ Matrícula: \\ BCC328 — Construção de Compiladores I — DECOM/UFOP}
\date{\today}

\begin{document}
\maketitle

\tableofcontents
\clearpage

% ==========================================================
\section{Entrega 1 — Etapa 1: Análise Léxica e Sintática (30/11/2025)}

\subsection{O que foi feito e o que não foi feito}

Todos os itens do trabalho 1 foram feitos

\subsubsection*{Foi feito}

\begin{itemize}
\item Definição de tokens com posição (linha e coluna), incluindo literais, operadores, palavras reservadas e símbolos da linguagem.
    \item Implementação do lexer com tratamento de erro léxico e exceção dedicada.
    \item Implementação do parser com precedência e associatividade para operadores lógicos, relacionais e aritméticos.
    \item Construção de AST com suporte a funções, structs, arrays, controle de fluxo, chamadas, tipos e inicializações.
    \item Interface de linha de comando para \texttt{--lexer}, \texttt{--parser} e \texttt{--pretty}.
    \item Testes automatizados para casos válidos e inválidos de análise léxica e sintática.
\end{itemize}

\subsection{Prompts utilizados, resultados e uso no trabalho}

Não possuo mais os prompts mas no geral, foi pra entender como funciona um lexer e um parser, e prompts gerais sobre a sintaxe do happy e do alex.

\subsection{Divisão de tarefas entre os membros}
Esta seção não se aplica, pois o trabalho foi realizado individualmente.

\clearpage

% ==========================================================
\section{Entrega 2 — Etapa 2: Análise Semântica e Interpretador (20/02/2026)}

\subsection{O que foi feito e o que não foi feito}

\subsubsection*{Foi feito}
\begin{itemize}
    \item Implementação de análise semântica.
    \item Implementação de tipos semânticos e unificação.
    \item Checagens de tipo em operações, condições, atribuições, retorno e chamadas.
    \item Verificações de escopo, duplicidade de declaração e acesso a campos/índices.
    \item Implementação do interpretador.
    \item Integração completa na CLI.
    \item Suíte de testes para semântica, interpretador e limitações esperadas.
\end{itemize}

\subsubsection*{Não foi feito}
\begin{itemize}
    \item Função de ordem superior.
\end{itemize}

\subsection{Prompts utilizados, resultados e uso no trabalho}

% Se não usou IA:
% "Não foram utilizados prompts/IA nesta entrega."

\subsubsection*{Prompt 1}
\begin{itemize}
  \item \textbf{Prompt:} ...
  \item \textbf{Resultado obtido:} ...
  \item \textbf{Como foi utilizado:} ...
\end{itemize}

\subsubsection*{Prompt 2}
\begin{itemize}
  \item \textbf{Prompt:} ...
  \item \textbf{Resultado obtido:} ...
  \item \textbf{Como foi utilizado:} ...
\end{itemize}

\subsection{Divisão de tarefas entre os membros}
Esta seção não se aplica, pois o trabalho foi realizado individualmente.